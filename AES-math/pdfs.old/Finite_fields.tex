\documentclass[11pt, oneside]{article} 
\usepackage{geometry}
\geometry{letterpaper} 
\usepackage{graphicx}
	
\usepackage{amssymb}
\usepackage{amsmath}
\usepackage{parskip}
\usepackage{color}
\usepackage{hyperref}

\graphicspath{{/Users/telliott_admin/Dropbox/Tex/png/}}
% \begin{center} \includegraphics [scale=0.4] {gauss3.png} \end{center}

\title{Finite fields}
\date{}

\begin{document}
\maketitle

\Large

This chapter is the first of a series documenting my fun with cryptography.  We're talking about finite fields and it will sound a bit math-wonky at times but most everything will be obvious once we've worked through it.  Two of my sources are

\url{https://engineering.purdue.edu/kak/compsec/Lectures.html}

\url{http://www.cs.utsa.edu/~wagner/laws/FFM.html}

Start with some notes from the first one, early chapters from Prof. Kak's course on cryptography.

\subsection*{Definitions}

A \textbf{group} is a set of objects plus a binary operation (operator o), with the following properties.  If a,b $\in$ G, then the operations exhibit:

\begin{itemize}
\item Closure:           $a \ o \ b = c$ $\Rightarrow$ c $\in$ G
\item Associativity:    $(a \ o \ b) \ o \ c = a \ o \ (b \ o \ c)$
\item Identity element:  $a \ o \ i = a$
\item Inverse element:   $a \ o \ b = i$
\end{itemize}

A common notation is to use \{G,+\}, (even if the operation is not really like addition).  If $a + i = a$, call $i$ the identity element and typically use $0$ for it.

An \textbf{Abelian group} is:
\begin{itemize}
\item Commutative:       $a \ o \ b = b \ o \ a$
\end{itemize}

A \textbf{Ring} is a group with the multiplication operator $\times$ (even if the operation is not really like multiplication).  It may be designated as \{R,+,$\times$\} and exhbits:
\begin{itemize}
\item Closure:           $a \times b \in R$
\item Associativity:     $(a \times b) \times c = a \times (b \times c)$
\item Distributivity:    $a \times (b + c) = (a \times c) + (a \times b)$
\end{itemize}

Often the $\times$ is dropped:  $a(b + c) = ac + ab$.  

A ring \emph{may} be 
\begin{itemize}
\item Commutative:    $ab = ba$
\end{itemize}

An \textbf{integral domain} \{R,+,$\times$\} is a commutative ring that also has
\begin{itemize}
\item Multiplicative identity element:    $a \times 1 = a$
\end{itemize}

If $ab = 0$, then either $a = 0$ or $b = 0$.

A \textbf{Field} \{F,+,$\times$\} is an integral domain that has, for every $a$ a multiplicative inverse $b$
\begin{itemize}
\item Multiplicative inverse:   $ab = 1$
\end{itemize}

$1$ is its own multiplicative inverse.

According to wikipedia

\url{https://en.wikipedia.org/wiki/Finite_field}

\begin{quote}In mathematics, a finite field or Galois field ... is a field that contains a finite number of elements. As with any field, a finite field is a set on which the operations of multiplication, addition, subtraction and division are defined and satisfy certain basic rules.\end{quote}

You can read all about it there.  I \emph{think} this conveys the general idea.  

\subsection*{Polynomial arithmetic}
We switch to a new topic.  The connection with fields will be apparent shortly.

A polynomial is an expression of the form:
\[ \sum_0^n a_n x^n \]
where the coefficients come from some set S, for example, the integers:
\[ x^5 + 9x^3 + 2x^2 + 1 \]
This is a polynomial \emph{of degree} 5.

Polynomial arithmetic deals with addition, multiplication, etc. of polynomials.  Consider this example of division for polynomials with cofactors from the real numbers:
\[ \frac{8x^2 + 3x + 2 }{2x + 1} \]

The first term of the quotient is $4x$ (because $4x \times 2x = 8x^2$) and 
\[ 4x \times (2x+1) = 8x^2 + 4x \]
so we subtract that from the numerator and the remainder is $-x + 2$ and dividing again
\[ \frac{-x + 2}{2x + 1} \]
The second term of the quotient is $-0.5$ (because $-0.5 \times 2 = -1$ and
\[ -0.5 \times (2x + 1) = -x - 0.5 \]
Subtracting $-0.5$ from $2$ leaves a remainder of $2.5$.

\subsection*{Additive and multiplicative inverses}

Now, suppose we start doing arithmetic with polynomials whose coefficients belong to a finite field.  Example:  $Z_7$ which can also be called $GF(7)$.

We construct such a field simply by doing all our arithmetic modulo $7$.  If a value is greater than or equal to 7, we divide by $7$ and set the value equal to the remainder.

We will be doing division and subtraction mod $7$.  For division that means finding a multiplicative inverse for the denominator and \emph{multiplying} the numerator by that.  Similarly, for subtraction we find the additive inverse of the second term and \emph{add} that to the first term.

\textbf{Additive inverses}
\[ 1 + 6 = 0 \  \text{mod} \  7 \]
\[ 2 + 5 = 0 \  \text{mod} \  7 \]
\[ 3 + 4 = 0 \  \text{mod} \  7 \]
So, for example, subtracting $3$ is the same as adding $4$

\textbf{Multiplicative inverses}.
\[ 1 \ \text{ is its own inverse} \]
\[ 2 \times 4 = 8  = 1 \  \text{mod} \  7 \]
\[ 3 \times 5 = 15 = 1 \  \text{mod} \  7 \]
\[ 6 \times 6 = 35 = 1 \  \text{mod} \  7 \]
so $6$ is also its own multiplicative inverse.

Example
\[ \frac{5x^2 + 4x + 6}{2x + 1} \]

We first divide $5$ by $2$.  Since $4$ is the multiplicative inverse of $2$ we multiply $5 \times 4 = 20 = 6  \  \text{mod} \ 7$.  So the first term of the quotient is $6x$ and
\[ 6x \times (2x + 1) = 5x^2 + 6x \]
We need to subtract $4x-6x$ which we do by adding $4x + 1x = 5x  \  \text{mod} \  7$.  Hence we now have the remainder
\[ \frac{5x + 6}{2x + 1} \]

We did this division $5/2$ before:  we got $6$.
\[ 6 \times (2x + 1) = 5x + 6 \]
which leaves no remainder.  The answer is $6x + 6$.

Hence we can write:
\[ (6x + 6)(2x + 1) = 5x^2 + 4x + 6 \]
That can be done pretty easily without paper.  Multiplication is definitely easier than division.

We call $(6x + 6)$ and $(2x + 1)$ the factors of $(5x^2 + 4x + 6)$.

\subsection*{GF(2)}

Now we're getting closer to the main point.  We will be doing binary arithmetic and the coefficients of the polynomials come only from $0$ and $1$.  Therefore, the polynomials are of the form
\[ \sum_0^n x^n \] 

There are no coefficients now.  Either a term is zero or it is $x$ to some power like $x^n$.

GF(2) consists of the set \{0,1\}.

We define \textbf{addition}
\[ 0 + 0 = 0 \]
\[ 0 + 1 = 1 \]
\[ 1 + 0 = 1 \]
\[ 1 + 1 = 0 \]

Addition is the same as logical XOR.

\textbf{subtraction}
\[ 0 - 0 = 0 \]
\[ 1 - 0 = 1 \] 
\[ 0 - 1 = 1 \]
\[ 1 - 1 = 0 \]

Notice here that (i) $1 - 1 = 0$ because (from the first table) the arithmetic inverse of $1$ is just $1$ since $1 \oplus 1 = 0$, so $1 - 1 = 1 + 1 = 0$.  For a similar reason $0 - 1 = 1$.

Another reason is that in moving  $0$ to $-1$ we move by one unit.

\textbf{multiplication}
\[ 0 \times 0 = 0 \]
\[ 0 \times 1 = 0 \]
\[ 1 \times 0 = 0 \]
\[ 1 \times 1 = 1 \]

Multiplication is the same as logical AND.

Let us work with two such polynomials:  
\[ (x^4 + x^3 + x + 1) \text{, and } (x^3 + x^2)  \]

Addition::
\[ (x^4 + x^3 + x + 1) + (x^3 + x^2)  \]

What to do with $2x^3$?  We do the addition mod $2$, and so obtain zero for the coefficient of $x^3$.  And the important rule is:  we \emph{do not} "carry the one."
\[ = x^4 + x^2 + x + 1 \]

Multiplication:
\[ (x^2 + x + 1)(x + 1) = x^3 + x^2 + x + x^2 + x + 1 \]
Here, there are two of $x^2$ and two of $x$ which all cancel.
\[ = x^3 + 1 \]

I have a question at this point, why don't we treat this as
\[ (0x^3 + 1x^2 + 1x + 1)(0x^3 + 0x^2 + 1x + 1) \]
and use the rules for multiplying $0$ and $1$ above?  In any event, we don't.

\textbf{division}
\[ \frac{x^2 + x + 1}{x + 1} \]
We can do this formally, or we can guess.  The formal method is to divide $x^2/x$ which is equal to $x$ so then
\[ x \times (x + 1) = x^2 + x \]
Subtraction gives $1$, and the answer is $x$ with a remainder of $1/x+1$.

When 
\[ \frac{f(x)}{g(x)} \] 
leaves no remainder, we say that $g(x)$ is a factor of $f(x)$ (and the quotient is another factor).

\subsection*{Irreducible polynomial}

An irreducible or prime polynomial is one without factors.  To restate this, to say that a given polynomial $p(x)$ is irreducible means that there do not exist:
\[ f(x) \times g(x) = p(x) \]

The set of polynomials over $GF(2)$ forms a ring, called the polynomial ring.

There are only two irreducible polynomials of degree 3 in $GF(2)$ and they are:
\[ x^3 + x + 1 \]
\[ x^3 + x^2 + 1 \]

It is claimed that you cannot find $f(x)$ and $g(x)$ such that $f(x) \times g(x)$ is equal to either of these, and these are the only polynomials in $GF(2)$ with that property.

Now that's a challenge.  Suppose we build up possible factors of these expressions, starting with  
\[ 1 \]
and continuing with polynomials with greatest term $x^1$.  There are two:
\[ x \]
\[ x + 1 \]
Now consider polynomials with greatest term $x^2$ formed by multiplying these last:
\[ x (x) = x^2 \]
\[ x (x + 1) = x^2 + x \]
\[ (x + 1)(x + 1) = x^2 + 1 \]

Now consider all products with greatest term $x^3$ by multiplying factors that we have generated so far:
\[ x (x^2) = x^3 \]
\[ x (x^2 + x) = x^3 + x^2 \]
\[ x (x^2 + 1) = x^3 + x \]
\[ (x + 1)(x^2) = x^3 + x^2 \]
\[ (x + 1)(x^2 + x) = x^3 + x \]
\[ (x + 1)(x^2 + 1) = x^3 + x^2 + x + 1 \]

And that's it.  There is no way to generate any other polynomial with greatest term $x^3$, and since these two do not appear in our results, there is no way to factor either $x^3 + x + 1$ or $x^3 + x^2 + 1$.

Proving something similar for higher degrees might be a challenge, but see Kak's proof, below.

\subsection*{Modulo an irreducible polynomial}
We will now consider all polynomials defined over $GF(2)$ modulo the irreducible polynomial $x^3 + x + 1$.

When multiplication results in a polynomial whose degree equals or exceeds that of the irreducible polynomial, we will take for our result the remainder modulo that polynomial.

Example:
\[ (x^2 + x + 1) \times (x^2 + 1) \mod x^3 + x + 1 \]
\[ = x^4 + x^2 + x^3 + x + x^2 + 1  \mod x^3 + x + 1  \]
\[ = x^4 + x^3 + x  + 1  \mod x^3 +x+1 \]

What is 
\[ \frac{x^4 + x^3 + x  + 1}{x^3 +x+1} \]
well
\[ x (x^3 + x + 1) = x^4 + x^2 + x \]
which when subtracted from the numerator leaves $x^3 - x^2 + 1$ so we have
\[ \frac{x^3 - x^2 + 1}{x^3 +x+1} \]
Now the quotient is $1$ with a remainder of $-x^2 - x$.

Recall that $-1 = 1$, because $1$ is its own additive inverse:  $1 + 1 = 0$ so $1 = 0 - 1$.  We have then $x^2 + x$.

Restate the result:
\[ \frac{x^4 + x^3 + x + 1}{x^3 + x + 1} = x + 1 + \frac{x^2 + x}{x^3 + x + 1} \]

Let's check:

\[ (x^3 +x+1) \times (x + 1) \]
\[ = x^4 + x^3 + x^2 + x + x + 1 \]
\[ = x^4 + x^3 + x^2 + 1 \]
which falls short of the original numerator $x^4 + x^3 + x  + 1$ by exactly $x^2 + x$.

There is a less error-prone way to do this kind of modulo operation and we will see it in the next chapter.

Polynomials defined over $GF(2)$ modulo the irreducible polynomial $x^3 + x + 1$ consist of the finite set:
\[ 0 \]
\[ 1 \]
\[ x \]
\[ x + 1 \]
\[ x^2 \]
\[ x^2 + 1 \]
\[ x^2 + x \]
\[ x^2 + x + 1 \]

It's starting to look familiar.

\[ 000 \]
\[ 001 \]
\[ 010 \]
\[ 011 \]
\[100 \]
\[101 \]
\[110 \]
\[111 \]


There are only eight of them.  We refer to this set as $GF(2^3)$.  3 is the degree of the modulus polynomial.

\end{document}